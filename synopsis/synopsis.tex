\documentclass[12pt,a4paper]{article}
\begin{document}
\begin{center}
分散コンピューティング基盤における 宣言的構成管理の適用に関する研究
\end{center}
\begin{flushright}
1930014~~真壁 徹
\end{flushright}
 クラウドコンピューティングをはじめとする分散コンピューティングの普及に伴い,基盤を構成するサーバやネットワークデバイス,ストレージなどのリソースは増加,多様化化している.そして,その運用は専門性を持つ技術者への依存を強めている.そこで,リソースの状態を意識する必要なく,また手続きを逐一指示せずとも,あるべき姿を定義すれば基盤をその通りに設定,維持する宣言的構成管理が注目されている.

宣言的構成管理は新しい概念ではなく,制約,論理プログラミングを基礎とした先行研究がある.しかし,管理対象の持つ自律性との齟齬,構成を決定する計算量と時間,リソース操作に要する時間,動作中のリソースを操作するリスク,制約,論理プログラミングを習得する難しさなど,多くの課題があった.

一方,オープンソースの基盤ソフトウェアであるKubernetesは,先行研究で指摘された宣言的構成管理の主要な課題を,結果整合を認めるなどトレードオフを受け入れて解決した.反面,普及の過程でアクシデントも散見され,安全性や回復性に関する負の影響も疑われる.

Kubernetes上で動作するアプリケーションの回復性については,コンテナの回復時間に注目した研究がある.しかし,それを支える構造は追究されていない.本研究はKubernetesを題材とし,その構造分析を通じ課題や論点,展望を示すことで,今後の宣言的構成管理の研究と実践へ貢献する.

Kubernetesは構成管理と回復機能を一体としている.回復機能は安全性を支える主要素である.よって,安全性分析を通じて,構成管理に関する特徴や洞察も得られると期待できる.本研究では手法にSTPA(System-Theoretic Accident Model and Processes)を選択し,構造と安全性を分析した.Kubernetesは多様な要素で構成されているため,構成要素の相互作用に注目するSTPAはその分析に適している.

本研究ではSTPAの手順に従い,まず全体とサブシステムの構造を分析するために制御構造図(コントロールストラクチャ)を作成し,非安全なコントロールアクションを導いた.次に,その原因であるハザードシナリオを得た.加えて,得たハザードシナリオでKubernetesの障害事例を分類し,その妥当性を評価した.

分析の成果物はKubernetesの利用者がその制御構造やハザードシナリオを理解する助けとなり,リスク評価や問題対処,カオスエンジニアリングに貢献する.加えて,分析から導いた論点は,Kubernetesに限らず宣言的構成管理の適用,活用の研究と実践において価値がある.中でも2つの特記すべき論点がある.

まず1つ目は,リソース作成,変更時の入力内容に対する妥当性検証の必要性である.ハザードシナリオと障害事例の分類において,妥当性検証の不足を原因とするアクシデントの数は顕著であった.例えば,リソース量の制限や優先度設定を行わずに投入されたアプリケーションが,リソースを共有する他のアプリケーションやシステム要素と競合し,ハザードに繋がっている.Kubernetesが結果整合を受け入れ,宣言の入力時検証を任意とした負の影響を認める.

2つ目の特記すべき論点は,アプリケーションの再構成耐性である.宣言的構成管理において構成管理機能は,あるべき姿を維持するため,主導的にリソースを再構成する.よってアプリケーションの再構成耐性は重要な論点である.障害が発生することを前提にアプリケーションを設計する「Design for Failure」という考えがあるが,宣言的構成管理においては,加えて再構成イベントも考慮すべきである.つまり障害や再構成を原因とした混乱状態に耐える設計「Design for Chaos」が求められる.実装例の1つにリクエストの再試行があるが,本研究では,その有無で再構成耐性に影響があるこを検証した.

アプリケーションの再構成耐性の確保は,検討を基盤の範囲に閉じていては解決できない.例えば,アプリケーションへ再試行機能などを実装するだけでなく,そのテストも課題となる.再構成耐性はアプリケーションを新規作成,変更時にテストするだけでは十分でない.リソース構成が変化していくことを考慮し,継続的にChaos testingを行うべきである.

今後はネットワークやストレージなど,現時点で宣言的構成管理の適用に課題があるリソースへと管理対象は広がるであろう.すでにクラウドサービス事業者では取り組みが始まっている.しかしリソース間の依存関係や変更時の影響の大きさなどから,変更不可能な属性を設けるなど制約はある.この制約の解決は将来の課題である.
\end{document}
